\documentclass[11pt]{article}
% \pagestyle{empty}

\setlength{\oddsidemargin}{-0.25 in}
\setlength{\evensidemargin}{-0.25 in}
\setlength{\topmargin}{-0.9 in}
\setlength{\textwidth}{7.0 in}
\setlength{\textheight}{9.0 in}
\setlength{\headsep}{0.75 in}
\setlength{\parindent}{0.3 in}
\setlength{\parskip}{0.1 in}
\usepackage{epsf}
\usepackage{pseudocode}
\usepackage{listings}
\usepackage{amsmath}

% \usepackage{times}
% \usepackage{mathptm}

\def\O{\mathop{\smash{O}}\nolimits}
\def\o{\mathop{\smash{o}}\nolimits}
\newcommand{\e}{{\rm e}}
\newcommand{\R}{{\bf R}}
\newcommand{\Z}{{\bf Z}}

\begin{document}
Programming Assignment 2 Ben Anandappa and Christi Balaki \newline

\textbf{Part 1}
\begin{itemize}
\item In standard matrix multiplication, multiplying two $n x n$ matrices by each other has the following costs:
\begin{itemize}
\item Multiplication: $n^3$ steps
\item Addition: $(n-1)n^2$ steps
\end{itemize}
Together, the total cost of the standard algorithm is $n^3 + (n-1)n^2 = (2n - 1)n^2$ 
\item In Strassen's method, we can calculate the total cost by first calculating the costs of $P_1$ to $P_7$ as described in the lecture notes and the costs of the additions/subtractions of these products. 

From the notes, we have the submatrices A, B, C, D, E, F, G, H which are of size n/2 and the following 7 products: 
\begin{itemize}
\item $P_1 = A(F-H)$
\item $P_2 = (A+B)H$
\item $P_3 = (C+D)E$
\item $P_4 = D(G-E)$
\item $P_5 = (A+D)(E+H)$
\item $P_6 = (B-D)(G+H)$
\item $P_7 = (A-C)(E+F)$
\end{itemize} 

of which we can use to make: 
\begin{itemize}
\item $AE + BG = P_5 + P_4 - P_2 + P_6$
\item $AF + BH = P_1 + P_2$
\item $CE + DG = P_3 + P_4$
\item $CF + DH = P_5 + P_1 - P_3 - P_7$
\end{itemize}

For a $n*n$ matrix, the cost C(n) can be

Note that the cost of $P_1$, $P_2$, $P_3$ and $P_4$ are the same ad

 \end{itemize}
\end{document}





